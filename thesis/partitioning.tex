\section{Partitioning}

\subsection{From hypergraph to graph}
Although we argued that going from a hypergraph to a graph had no incidence on the description of the system, there might be a consideration we overlooked.
The way it works is simply creating new nets for all pairs of cells in a hyperedge, with a length equal to the length of the hyperedge.
However, all the gates are not necessarily close to each other in a hyperedge, there might be some dispersion.
This remark lead to the dispersion clustering in section~\ref{sec:disp-clust}.

\subsection{Net dispersion and unbalance}\label{sec:part-disp}
When we cut a long net, we should be happy, a long connection being made shorter.
But should we?

A long net can mean several things:
\begin{itemize}
	\item Two cells far away from each other. The ideal case, that is exactly the type of net we want to cut. However, those à very rare.
	\item Several cells interconnected with a high disperion (see~\ref{sec:disp-clust}).
	Idealy, we would want to separate the isolated cell from the rest.
	But if we cut in the pack of gates close to each other, this might degrade the performance of the resulting 3D design.
	\item Lots of cells closely interconnected.
	This net should actually not be cut.
	This type of problem could be alleviated by dividing the wire-length of a net by the amount of interconnect gates. But again, this might ahe the disadvantage to put forward short nets with only two gates.
\end{itemize}


We could compute the dispersion before and after partitioning.
If a dispersion has increased for a given net, it means we did not cut it correctly and probably made the situation worse.

\subsection{MIN-CUT}

\subsection{MAX-CUT}

\subsection{Metrics}