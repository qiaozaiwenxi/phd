\documentclass[12pt,a4paper]{article}
\usepackage[utf8]{inputenc}
\usepackage[english]{babel}
\usepackage[T1]{fontenc}
\usepackage{amsmath}
\usepackage{amsfonts}
\usepackage{amssymb}
\usepackage{graphicx}
\usepackage[breaklinks=true,hidelinks]{hyperref}
% \usepackage[breaklinks=true]{hyperref}
% \usepackage[xindy, toc, nonumberlist]{glossaries}
\usepackage[toc, nonumberlist]{glossaries}
% \gls{label} to invoke the entry.
% \glsreset{label} to reset the first occurence of the entry.
% \acrlong{label} to print the long version of the acronym.
% \acrshort{label} to print the short version of the acronym.

\usepackage{todonotes}

\usepackage{etoolbox}
\AtBeginEnvironment{quote}{\itshape} 	% https://tex.stackexchange.com/questions/235852/how-to-make-all-quotes-italicised
										% 

% \usepackage[]{natbib}
\usepackage[numbers]{natbib}
\usepackage{amsthm}%theorems
\newtheoremstyle{customdef}%
	{\topsep}% above the theorem
	{\topsep}% below the theorem
	{\itshape}% body
	{0pt}% indent
	{\bfseries}% head
	{\newline}% punctuation between head and body
	{ }% Space after theorem head
	{\thmname{#1}\thmnumber{ #2}: \thmnote{#3}}
\theoremstyle{customdef}
\newtheorem{definition}{Definition}


\makeglossaries
\newglossaryentry{gl:netlist}
{
	name=netlist,
	description={File descripting the connectivity of an architecture.
				Such a list can either be \textit{pin-oriented} or \textit{net-oriented}.
				The former focuses on the design components or their associated nets, the later on the nets and their associated components.
				\newline
				A netlist can either be hierarchical or flat.}
}

\newglossaryentry{gl:primitive}
{
	name=primitive,
	description={In a netlist, a primitive is a definition which does not include any instance.
				As such, a flat netlist only contains primitives}
}

\newacronym{ac:rtl}{RTL}{Register Transfer Level}

\newglossaryentry{gl:rtl}
{
	name=register transfer level,
	description={Level at which are described the signals and the values they should take}
}

\newacronym{sa}{SA}{Simulated Annealing}
\newacronym{ts}{TS}{Tabu Search}

\author{Quentin Delhaye}
% \date{October 25th, 2015}&
\title{Automated System Partitioning for Efficient 3D Circuit Integration \\ \vspace{.5em} \hrule \vspace{.5em} \large Reaserch Progress Report}

\begin{document}
\maketitle
\tableofcontents

\clearpage

\begin{quote}
Imagine you are an antic civil contractor asked to handle high density parking for horse drawn caravans.
You have a fixed area terrain on which you can fit so much cars.
Along the years, they become smaller and less bulkier, but that scaling tend to slow down and you still need to fit more and more of them on your parking.
You now have to make a choice: Do you spend time and money to engineer expensive cars using new materials, or do you finally add a second floor to your parking?
\end{quote}

\section{Motivation: Why do we want to go 3D?}
\subsection{Why is 2D not enough anymore?}
\begin{itemize}
	\item MONEY
	\item Gate capacitance increasing
\end{itemize}


\section{State of the Art: How do we go 3D?}
\subsection{Hypergraph partitioning}
\subsubsection{Multilevel paradigm}
\subsubsection{Clustering}
\subsubsection{Implementations}
Numerous software libraries implement some flavor of (hyper)graph partitioning.
\begin{itemize}
	\item PaToH (\citet{Aykanat2011})
	\item hMetis (\citet{Karypis1999})
	\item Scotch (\citet{Aykanat2011})
	\item MLPart (\citet{Caldwell2000})
	\item Parkway (\citet{Trifunovic2008})
	\item Zoltan (\citet{Devine2006}
)	\item Chaco (\citet{Lotfifar2015})
	\item Jostle (\citet{Walshaw1998})
	\item Party (\citet{Preis97party})
	\item KaFFPa (\citet{Holtgrewe2010})
\end{itemize}
\subsubsection{Why hypergraphs?}
All hypergraphs can be represented as regular graphs, and lots of tools already exists to handle graphs.
When we want to express the wire length between two nodes, although it can easily be done using graph's edge weights, it is not expressable as a hypergraph's hyperedge weight, the hyperedge linking several nodes all connected with different wire lengths.

However, when tackling the partitioning problem, working with hypergraphs gives the vertices belonging to the same hyperedge a coherence.
In the context of 3D ICs, we work with buses connecting blocks of gates, and we want them to be handled as such.

Moreover, \citet{Ihler1993} demonstrated that the mincut partition obtained for a circuit is not as accurate as would be a hypergraph's.
More precisely, they define a \textit{cut-model} in definitions~\ref{def:cut-model} and \ref{def:cut-model-formal}, and show that there is no such thing in general.

\begin{definition}[Cut-model]\label{def:cut-model}
An edge-weighted graph $(V,E)$ is a cut-model for an edge-weighted hypergraph $(V,H)$ if the weight of the edges cut by any bipartition of $V$ in the graph is the same as the weight of the hyperedges cut by the same bipartition in the hypergraph.
\end{definition}

A more formal way to define the principle is as follows~:
\begin{definition}[Cut-model and mincut-model]\label{def:cut-model-formal}
A graph $(V, E)$ on $k$ vertices is a cut-model (for a unit weight hyperedge on $k$ vertices) if the weight of any cut induced by a non-empty proper subset $W$ of $V$ is equal to one.

A graph $(V \cup D,E)$ on $k+d$ vertices is called a min-cut-model (for a unit weight hyperedge on $k$ vertices) if for every non-empty subset $W$ of $V$ we have that the weight of any cut with minimum weight (mincut) under those separating $W$ from $V \setminus W$ is equal to one.
It must be zero fo $W=\emptyset$.%Note: other symbol for empty set: \varnothing
\end{definition}

\subsection{Monolithic vs stacked}
\subsection{Memory-on-logic vs logic-on-logic}


\section{Challenges: Why is it not done yet?}
\subsection{Heat Dissipation}
\subsection{Manufacturing}
\subsection{Optimal Partitioning}


\section{Work: Where are we now?}
\subsection{Graph Extraction}
\subsection{Naïve Clustering}
\subsection{Stats extraction}
\subsection{Interface with DEF--PaToH/hMetis--Cadence}


\section{Results: What can we say?}
\subsection{WL}
\subsection{Asymetric partitions}
\subsection{Power aware}

\section{Future}
\subsection{Verilog partitioning}
\subsection{Accessibility/UI}
\subsection{Clustering}
What you need to keep in mind, it's that the multilevel paradigm already clusterizes the vertices of the graph during the coarsening phase.
The purpose of the naïve clustering we are applying on the design, is (1) to reduce the order of the graph before feeding it to the partitioner, and (2) maintain the integrity of logical blocks, hence easing the subsequent \gls{pr} of the partitioned dies.
At the moment, we do not yet need a very efficient clustering for the current purpose.
A future improvement could be to develop a more sophisticated algorithm that could replace the coarsening and uncoarsening of the multilevel paradigm, allowing us to fall back on a raw graph partitioning algorithm.
\subsection{New metrics}
\subsection{Multi-die partitioning}

\todo[inline]{I do not intend to develop a new partitioning method: the aim is not to develop yet another derivation of an existing algo.
I want to focus on a usable worklow interfacable with production tools.}
\clearpage

\glsaddall
\printglossaries

\newpage
\bibliographystyle{abbrvnat}
% \bibliographystyle{acm}
\bibliography{../thesis/bibliography.bib}

\end{document}
