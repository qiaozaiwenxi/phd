\documentclass[xcolor={x11names, rgb, usenames, dvipsnames}]{beamer}
% Beamer loads xcolor by default. Do not load it a second time using \usepackage

\usepackage[francais, english]{babel}
\usepackage[T1]{fontenc}
\usepackage[utf8]{inputenc}
\usepackage{pgfplots}
% \pgfplotsset{compat=1.12}
\usetikzlibrary{patterns}
\usepackage{graphicx}
\usepackage{hyperref}
\usepackage{amsmath}
\usepackage{amssymb}
\usepackage{tabularx}
\usepackage{multirow}
\usepackage{array}


\setbeamertemplate{bibliography item}{[\theenumiv]}

%\usetheme{Warsaw}
% \usetheme{Boadilla}
% \usetheme{Antibes}
\usetheme{CambridgeUS}
% \usecolortheme{dolphin}
\usecolortheme{dolphin}
% \usetheme{Berlin}
% \usetheme{Madrid}
% \setbeamertemplate{footline}[frame number]


%http://tex.stackexchange.com/questions/160825/modifying-margins-for-one-slide
\newcommand\Wider[2][3em]{%
\makebox[\linewidth][c]{%
  \begin{minipage}{\dimexpr\textwidth+#1\relax}
  \raggedright#2
  \end{minipage}%
  }%
}




% http://tex.stackexchange.com/questions/116077/presentation-beamer-title-page
\makeatletter
\newcommand\titlegraphicii[1]{\def\inserttitlegraphicii{#1}}
\titlegraphicii{}
\setbeamertemplate{title page}
{
  \vbox{}
  \vspace{-2.5em}
   {\usebeamercolor[fg]{titlegraphic}\inserttitlegraphic\hfill\inserttitlegraphicii\par}
  \vskip2.5em
  \begin{centering}
    \begin{beamercolorbox}[sep=8pt,center]{institute}
      \usebeamerfont{institute}\insertinstitute
    \end{beamercolorbox}
    \begin{beamercolorbox}[sep=8pt,center]{title}
      \usebeamerfont{title}\inserttitle\par%
      \ifx\insertsubtitle\@empty%
      \else%
        \vskip0.25em%
        {\usebeamerfont{subtitle}\usebeamercolor[fg]{subtitle}\insertsubtitle\par}%
      \fi%
    \end{beamercolorbox}%
    \vskip1em\par
    \begin{beamercolorbox}[sep=8pt,center]{date}
      \usebeamerfont{date}\insertdate
    \end{beamercolorbox}%\vskip0.5em
    \begin{beamercolorbox}[sep=8pt,center]{author}
      \usebeamerfont{author}\insertauthor
    \end{beamercolorbox}
  \end{centering}
  %\vfill
}
\makeatother

\author{Quentin Delhaye}
\title[Partitioning for 3D stacking]{Automated System Partitioning \\ for Efficient 3D Circuit Integration \\\rule{2cm}{0.3pt} ~\\ Work presentation}
% \subtitle{}
\institute[BEAMS]{Université Libre de Bruxelles}
\date{October 30th, 2015}

% \titlegraphic{\includegraphics[width=1.5cm]{ulbnorm}}
% \titlegraphicii{\includegraphics[width=1.5cm]{logo-polytech-seul}}


%%%%%%%%%%%%%%%%%%%%%%%%
% data
%%%%%%%%%%%%%%%%%%%%%%%%
\def\temperaturedata{temperaturesOslo.txt}
\tikzstyle{maxmark} = [mark=*,mark options={color=red,scale=15}]
\tikzstyle{minmark} = [mark=*,mark options={color=blue,scale=15}]

% Lists
% \def\labelitemi{$\blacktriangleright$}

% \AtBeginSection[]
% {
%   \begin{frame}
%   \frametitle{Contents}
%   \tableofcontents[currentsection]
%   \end{frame}
% }


\begin{document}

\begin{frame}[plain, noframenumbering]
\titlepage
\end{frame}

\begin{frame}
	\frametitle{Contents}
	\tableofcontents[hideallsubsections]
	% \tableofcontents
\end{frame}

% \section{Presentation}
% \begin{frame}
%   \frametitle{Presentation}
%   \begin{itemize}
%     \item Master in Computer Science and Engineering (2015)
%     \item Teaching assistant in applied electronics, real-time systems, etc.
%   \end{itemize}
% \end{frame}


\section{Context}

\begin{frame}
\frametitle{Context}
\begin{itemize}
	\item Partition integrated circuits and stack the partitions in 3D
  \item Find clever ways to set the weights of the equivalent graph
\end{itemize}
\end{frame}

\section{Work}
\begin{frame}
  \frametitle{Work done so far}
  \begin{itemize}
    \item Using metis and hmetis to partition architectures (MPSoC, OpenMSP, RTX, SPC)
    \item Considerations: Area, power, wire length, number of connections, etc.
    \item Attempts at asymmetrical partitioning
  \end{itemize}
\end{frame}

\section{Focus}
\begin{frame}
  \frametitle{Focus}
  \begin{itemize}
    \item Maxcut algorithms (linear programming, semidefinite programming, heuristics)
    \item Graph partitioning methods (single and multiweighted)
    \item Explore existing implementations of the above (Metis, Party, Scotch, etc.)
  \end{itemize}
\end{frame}





%%%%%%%%%%%%%%%%%%%%%%%%%%%%%%%%%%%%%%%%%%%%%%
% 				BACKUP SLIDES
%%%%%%%%%%%%%%%%%%%%%%%%%%%%%%%%%%%%%%%%%%%%%%


% \begin{frame}[noframenumbering]
% \frametitle{Software GCM}
% 	\begin{figure}
% 	\input{ipsec-ping-benchmark-gcm}
% 	\caption{Software: asm kernel module mode GCM\newline{} Hardware: AES IP core mode CBC}
% 	\end{figure}
% \end{frame}




% %%%%%%%%%%%%%%%%%%%%%%%%%%%%%%%%%%%%%%%%%%%%
% \section*{References}
% %%%%%%%%%%%%%%%%%%%%%%%%%%%%%%%%%%%%%%%%%%%%

% \nocite*{}
% \bibliographystyle{plain}
% \bibliography{bibliography}

\end{document}
