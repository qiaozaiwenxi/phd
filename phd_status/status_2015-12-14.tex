\documentclass[xcolor={x11names, rgb, usenames, dvipsnames}]{beamer}
% Beamer loads xcolor by default. Do not load it a second time using \usepackage

\usepackage[francais, english]{babel}
\usepackage[T1]{fontenc}
\usepackage[utf8]{inputenc}
\usepackage{pgfplots}
\pgfplotsset{compat=newest}
\usetikzlibrary{patterns}
\usepackage{graphicx}
\usepackage{hyperref}
\usepackage{amsmath}
\usepackage{amssymb}
\usepackage{tabularx}
\usepackage{multirow}
\usepackage{array}
% \usepackage{enumitem}
\setlength{\leftmargini}{1em}%Shift the \item to the left.

\setbeamertemplate{navigation symbols}{}
\setbeamertemplate{bibliography item}{[\theenumiv]}

\usetheme{CambridgeUS}
\usecolortheme{dolphin}


\begin{document}

\begin{frame}[plain, noframenumbering]
  \frametitle{Progress (2015-12-14)}
  \fbox{
    \begin{minipage}[b][0.9\textheight][t]{0.45\linewidth}
      Status
      \begin{itemize}
        \item Heuristics are mostly used in VLSI because they require less computing power.
        Partitioning problems are clustered to be solved in a realistic amount of time, then the solution is refined by uncoarsening the graph.
        \item Meta-heuristics in VLSI are barely a few years old.
        \item F-M heuritic is designed to be multi-weight and balance constrained, but not always fully implemented.
      \end{itemize}
      % \vspace*{0.8\textheight}
      \hfill
    \end{minipage}
  }
  \begin{minipage}[b][0.9\textheight][t]{0.45\linewidth}%

    %%%%%%%%%%%%%%%%%%
    % Highlights box
    %%%%%%%%%%%%%%%%%%
    \fbox{
      \begin{minipage}[b][0.15\textheight][t]{\linewidth}
        Highlights \& Low points
        \begin{itemize}
          \item Nothing particular
        \end{itemize}
      \end{minipage}
    }

    %%%%%%%%%%%%%%%%%%
    % TO-DO-box
    %%%%%%%%%%%%%%%%%%
    \fbox{
      \begin{minipage}[b][0.50\textheight][t]{\linewidth}
        TODO
        \begin{itemize}
          \item Check the heuristics used by the most popular soft implementations
          \item Find a complete implementation of F-M, or find an open source partial one, or develop a new one.
        \end{itemize}
      \end{minipage}
    }

    %%%%%%%%%%%%%%%%%%
    % Deliverables
    %%%%%%%%%%%%%%%%%%
    \begin{minipage}[b][0.25\textheight][t]{\linewidth}
      \footnotesize
      \begin{tabularx}{1.1\linewidth}{|X|X|X|l|} \hline
        & Planned & Due Date & Status \\ \hline
        Slide deck & Near future & 23/11 & In progress\\ \hline
      \end{tabularx}
    \end{minipage}

  \end{minipage}
\end{frame}

\end{document}
